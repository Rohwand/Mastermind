\documentclass[a4paper,12pt]{article}
\usepackage[french]{babel}
\usepackage[utf8]{inputenc}
\usepackage{graphicx}
\usepackage{xcolor}

\title{Mastermind - Paradigmes de programmation}
\author{Robin CHARRE}

\begin{document}

\maketitle

\section{Introduction}
Ce projet est une adaptation numérique du jeu \textbf{Mastermind}, où le joueur doit deviner une séquence de couleurs.  
L’objectif est de proposer une expérience interactive en HTML, CSS et JavaScript, avec une gestion des scores via un backend Node.js.  

\section{Règles du jeu}
Le joueur doit deviner une combinaison secrète de couleurs.  
\begin{itemize}
    \item La combinaison est composée de 4 couleurs choisies aléatoirement parmi une palette.
    \item Le joueur a \textbf{20 essais} pour trouver la bonne séquence.
    \item À chaque tentative, le jeu indique :
    \begin{itemize}
        \item combien de couleurs sont bien placées (\textbf{bonne couleur, bonne position}).
        \item combien de couleurs sont présentes mais mal placées (\textbf{bonne couleur, mauvaise position}).
    \end{itemize}
    \item Une fois la bonne combinaison trouvée, le joueur peut enregistrer son score.
    \item Seul le meilleur score d’un joueur est sauvegardé.
\end{itemize}

\section{Développement du projet}
Le projet a été développé en combinant plusieurs technologies modernes :

\subsection{Interface utilisateur (HTML, CSS)}
L’interface comprend :
\begin{itemize}
    \item Une grille pour sélectionner les couleurs.
    \item Un bouton de validation des essais.
    \item Un affichage des indices après chaque tentative.
    \item Une zone d’enregistrement du pseudo en fin de partie.
\end{itemize}

Le style est géré via \textbf{CSS} pour offrir une expérience visuelle agréable.

\subsection{Logique du jeu (JavaScript)}
\begin{itemize}
    \item Génération d’une combinaison de couleurs aléatoire.
    \item Vérification des essais du joueur.
    \item Affichage des indices sur la bonne combinaison.
    \item Gestion du score.
\end{itemize}
\end{document}
